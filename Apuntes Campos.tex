\documentclass{article}
\usepackage[utf8]{inputenc}
\usepackage[spanish,es-nodecimaldot]{babel}
\usepackage{amsmath}
\usepackage{amssymb}
\usepackage{amsthm}
\usepackage{mathtools}
\usepackage[bottom]{footmisc}
\usepackage{xcolor}
\DeclarePairedDelimiter{\ceil}{\lceil}{\rceil}
\usepackage{graphicx}
\renewcommand\qedsymbol{$\blacksquare$}
\usepackage{enumitem}
\newcommand*{\vcenteredhbox}[1]{\begin{tabular}{@{}c@{}}#1\end{tabular}}
\newcommand*{\colorboxed}{}
\usepackage{physics}

\DeclareMathOperator{\interior}{int}
\DeclareMathOperator{\osc}{osc}
\DeclareMathOperator{\gr}{gr}


\def\colorboxed#1#{%
  \colorboxedAux{#1}%
}
\newcommand*{\colorboxedAux}[3]{%
  % #1: optional argument for color model
  % #2: color specification
  % #3: formula
  \begingroup
    \colorlet{cb@saved}{.}%
    \color#1{#2}%
    \boxed{%
      \color{cb@saved}%
      #3%
    }%
  \endgroup
}

\usepackage{fancyhdr}
\usepackage[left=1.2in,right=1.2in,top=1in,bottom=1.25in,%
            footskip=.25in]{geometry}
\setlength\parindent{0pt}

\newcommand\blfootnote[1]{%
  \begingroup
  \renewcommand\thefootnote{}\footnote{#1}%
  \addtocounter{footnote}{-1}%
  \endgroup
}

\title{Álgebra Moderna 2: Teoría de Campos}
\author{\Large Rafael Dubois\\ Universidad del Valle de Guatemala \\ \texttt{dub19093@uvg.edu.gt}}
\date{\today}

\pagestyle{fancy}
\fancyhf{}
\renewcommand{\headrulewidth}{2pt}
\fancyfoot{}
\rhead{\footnotesize Teoría de Campos}
\lhead{\footnotesize Álgebra Moderna 2}
\rfoot{\thepage}
\lfoot{}
\setlength{\headheight}{28pt}

\begin{document}

\maketitle
\lhead{\footnotesize Universidad del Valle de Guatemala \\ 
\footnotesize Departamento de Matemática \\ 
\footnotesize Álgebra Moderna 2}
\rhead{\footnotesize Licenciatura en Matemática Aplicada \\ 
\footnotesize Rafael Dubois \\ 
\footnotesize Carné 19093}
\thispagestyle{fancy}

\section{Campos de extensión}

Un conjunto no vacío $F$ es un campo si en él se definen dos operaciones binarias (a menudo llamadas suma ($+$) la primera, y producto ($\cdot$) la segunda), las cuales cumplen con que $(F,+)$ y $(F-\{0\},\cdot)$ son grupos abelianos, y el producto se distribuye sobre la suma.

\subsection*{\color{violet} Extensión y subcampos}

Si $F$ es un campo y $K$ también es un campo tal que $F\subseteq K$, se dice que $F$ es subcampo de $K$ y que $K$ es una extensión de $F$.

\subsection*{\color{violet} Grado de una extensión}

El grado $[K:F]$ de una extensión $K$ de un campo $F$ es la dimensión de $K$ como espacio vectorial sobre $F$. Cuando $K$ es una extensión finita de $F$, se tiene $[K:F]\in\mathbb{Z}^{+}$.

\subsection*{\color{red} Teorema 5A:}

Si $L$ es una extensión finita del campo $K$, y $K$ es una extensión finita del campo $F$, entonces $L$ es una extensión finita de $F$ y $[L:F]=[L:K][K:F]$.

\subsubsection*{\color{red} Corolario 1 del teorema 5A:}

Si $L$ es una extensión finita del campo $F$ y $K$ es un subcampo de $L$ que contiene a $F$ (es decir, $F\subseteq K\subseteq L$), entonces $[K:F]\mid[L:F]$.

\subsubsection*{\color{red} Corolario 2 del teorema 5A:}

Si $F$ es un campo, $K$ es una extensión finita de $F$ y $[K:F]$ es un número primo, entonces no existe ningún campo $L$ tal que $F\subset L\subset K$.

\subsection*{\color{violet} Elementos algebraicos}

Si $F$ es un campo y $K$ es una extensión de $F$, entonces $a\in K$ es algebraico sobre $F$ si existe $f(x)\in F[x]$ un polinomio no constante tal que $f(a)=0$. Es decir, $a$ es una raíz de $f$.

\blfootnote{En este documento, los códigos de color van de la siguiente manera: Negro, Títulos; \color{blue} Azul, Lemas; \color{red} Rojo, Teoremas; \color{violet} Violeta, Definiciones; \color{purple} Morado, Propiedades; \color{teal} Aqua, Otros subtítulos.}

\newpage
\pagestyle{fancy}
\fancyhf{}
\renewcommand{\headrulewidth}{2pt}
\fancyfoot{}
\rhead{\footnotesize Teoría de Campos}
\lhead{\footnotesize Álgebra Moderna 2}
\rfoot{\thepage}
\lfoot{}
\setlength{\headheight}{28pt}

\subsection*{\color{violet} Valor de un elemento en un polinomio}

Si $F$ es un campo, $K$ es una extensión de $F$, $f(x)\in F[x]$ y $a\in K$, entonces $f(a)$ es el valor de $f$ en $a$, y si $f(a)=0$ se dice que $a$ satisface a $f$.

\subsection*{\color{purple} Colecciones de subcampos con elementos específicos}

Sea $F$ un campo, $K$ una extensión de $F$, $a\in K$ y $M=\{L: \text{$L$ es subcampo de $K$, $a\in L$ y $F\subseteq L$}\}$. Entonces $M\neq\emptyset$, y si $F(a)=\displaystyle\bigcap_{L\in M} L$, se tiene $F(a)\in M$.

\subsubsection*{\color{teal} El subcampo más pequeño que contiene a un elemento}

En el orden parcial de la contención, $F(a)$ es el subcampo más pequeño de $K$ tal que $a\in F(a)$ y $F\subseteq F(a)$. Es decir, $F(a)$ es la extensión más pequeña de $F$ que contiene a $a$. 

\subsection*{\color{violet} Subcampo por adjunción}

Si $F$ es un campo, $K$ una extensión de $F$ y $a\in K$, entonces $F(a)$ es el subcampo de $K$ obtenido por adjunción de $a$ a $F$.
 
\subsection*{\color{purple} El subcampo por adjunción es de funciones racionales}

Si $F$ es un campo, $K$ una extensión de $F$ y $a\in K$, entonces $F(a)=\left\{\displaystyle\frac{p(a)}{q(a)}: p(x),q(x)\in F[x], q(x)\neq0\right\}$.

\subsection*{\color{red} Teorema 5B:}

Si $F$ es un campo y $K$ es una extensión de $F$, entonces $a\in K$ es algebraico sobre $F$ si y solo si $F(a)$ es una extensión finita de $F$.

\subsection*{\color{violet} Grado de un elemento algebraico}

Si $F$ es un campo, $K$ es una extensión de $F$ y $a\in K$, entonces $a$ es algebraico de grado $n$ sobre $F$ si el grado mínimo de un polinomio en $F[x]$ satisfecho por $a$ es $n$.

\subsection*{\color{red} Teorema 5C:}

Para $F$ un campo y $K$ una extensión de $F$, $a\in K$ es un elemento algebraico de grado $n$ sobre $F$ si y solo si $[F(a):F]=n$.

\subsection*{\color{red} Teorema 5D:}

Si $F$ es un campo, $K$ es una extensión de $F$ y $a,b\in K$ son algebraicos sobre $F$, entonces todos los elementos $a\pm b$, $ab$ y $a/b$ (si $b\neq0$) son algebraicos sobre $F$. Es decir, el conjunto de elementos algebraicos de $K$ sobre $F$ forma un subcampo de $K$.

\subsubsection*{\color{red} Corolario del teorema 5D:}

Si $F$ es un campo, $K$ es una extensión de $F$ y $a,b\in K$ son algebraicos de grados $m$ y $n$ sobre $F$, entonces todos los elementos $a\pm b$, $ab$ y $a/b$ (si $b\neq0$) son algebraicos de grados a lo más $mn$ sobre $F$.

\subsection*{\color{teal} Notación para subcampos por adjunción doble}

Si $F$ es un campo, $K$ es una extensión de $F$ y $a,b\in K$, entonces $[F(a)](b)=F(a,b)$.

\subsection*{\color{purple} Simetría de la doble adjunción}

Si $F$ es un campo, $K$ es una extensión de $F$ y $a,b\in K$, entonces $F(a,b)=F(b,a)$.

\subsection*{\color{teal} Notación para subcampos por adjunción múltiple}

Si $F$ es un campo, $K$ es una extensión de $F$ y $a_1,ldots,a_n\in K$, entonces $F(a_1,\ldots,a_n)$ es el subcampo más pequeño de $K$ que contiene a $F$ y a $a_1,ldots,a_n$.

\subsection*{\color{violet} Extensión algebraica}

Si $F$ es un campo y $K$ es una extensión de $F$ cuyos elementos son todos algebraicos sobre $F$, entonces $K$ es llamada una extensión algebraica.

\subsection*{\color{purple} Un ejemplo general de extensiones algebraicas}

Si $F$ es un campo, $K$ es una extensión de $F$ y $a\in K$, entonces $F(a)$ es una extensión algebraica.

\subsection*{\color{red} Teorema 5E:}

Si $L$ es una extensión algebraica de un campo $K$, el cual es extensión algebraica de un campo $F$, entonces $L$ es extensión algebraica de $F$.

\subsection*{\color{purple} Números complejos algebraicos}

Un elemento del conjunto de los números complejos $\mathbb{C}$ es algebraico si este es algebraico sobre $\mathbb{Q}$.

\subsection*{\color{purple} Números complejos trascendentes}

Un elemento del conjunto de los números complejos $\mathbb{C}$ es trascendente si este no es algebraico.

\section{La trascendencia de $e$}

\subsection*{\color{red} Teorema 5F}

El número $e$, el cual es el resultado de evaluar en $x=1$ la función $f:\mathbb{R}\to\mathbb{R}$ con la propiedad de que para todo $x\in\mathbb{R}$ se cumple $f'(x)=f(x)$ y $f(0)=1$, es un número trascendente.

\newpage
\section{Raíces de polinomios}

\subsection*{\color{violet} Raíz de un polinomio}

Si $F$ es un campo con $p(x)\in F[x]$, y $K$ es una extensión de $F$, entonces $a\in K$ es una raíz de $p(x)$ si $p(a)=0$.

\subsection*{\color{blue} Lema 5.1 (Teorema del residuo):}

Si $F$ es un campo con $p(x)\in F[x]$, y $K$ es una extensión de $F$, entonces para todo $k\in K$ se tiene que $p(x)=(x-k)q(x)+p(k)$, donde $q(x)\in K[x]$ y $\gr(q)=\gr(p)-1$.

\subsubsection*{\color{blue} Corolario del lema 5.1:}

Si $F$ es un campo, $K$ es una extensión de $F$ y $a\in K$ es una raíz de $p(x)\in F[x]$, entonces $x-a\mid p(x)$.

\subsection*{\color{violet} Raíz de multiplicidad $m$}

Si $F$ es un campo con $p(x)\in F[x]$, y $K$ es una extensión de $F$, entonces $a\in K$ es una raíz de multiplicidad $m\in\mathbb{Z}^+$ para $p(x)$ si $(x-a)^m\mid p(x)$ pero $(x-a)^{m+1}\nmid p(x)$.

\subsection*{\color{blue} Lema 5.2:}

Un polinomio de grado $n$ sobre un campo tiene a lo más $n$ raíces en cualquier extensión del campo, contando $m$ raíces para una raíz de multiplicidad $m$.

\subsection*{\color{red} Teorema 5G:}

Si $F$ es un campo con $p(x)\in F[x]$ irreducible sobre $F$ con $\gr(p)\geq 1$, entonces existe una extensión $E$ de $F$ tal que $[E:F]=\gr(p)$, y $E$ contiene una raíz de $p(x)$.

\subsubsection*{\color{red} Corolario del teorema 5G:}

Si $F$ es un campo con $f(x)\in F[x]$, entonces existe una extensión finita $E$ de $F$ tal que esta contiene una raíz de $f(x)$ y $[E:F]\leq \gr(f)$. 

\subsection*{\color{violet} Campo de descomposición}

Si $F$ es un campo con $f(x)\in F[x]$, una extensión finita $E$ de $F$ es un campo de descomposición de $f(x)$ sobre $F$ si $f(x)$ puede descomponerse o factorizarse como el producto de polinomios lineales sobre $E$, pero en ningún subcampo propio de $E$. 

\subsection*{\color{red} Teorema 5H (Existencia de los campos de descomposición):}

Si $F$ es un campo con $f(x)\in F[x]$ tal que $\gr(f)\geq 1$, entonces existe $E$ extensión finita de $F$ con $[E:F]\leq \gr(f)!$ tal que $E$ contiene $\gr(f)$ raíces de $f(x)$. Es decir, $E$ contiene un juego completo de raíces de $f(x)$.	Cuando $[E:F]$ es mínimo, esto implica que los campos de descomposición existen.

\subsubsection*{\color{teal} Comentarios del teorema 5H:}

Más adelante, en la introducción a la teoría de Galois, se verá que existe un campo $F$ con $f(x)\in F[x]$ tal que el campo de descomposición $E$ para $f(x)$ sobre $F$ cumple $
[E:F]=\gr(f)!$. ¿Serán únicos los campos de descomposición?

\subsection*{\color{blue} Lema 5.3:}

Si $F$ y $F'$ son campos y $\tau:F\to F'$ con $\tau(\alpha)=\alpha'$ es un isomorfismo, entonces $\tau^{*}:F[x]\to F'[t]$ con $\displaystyle\tau^{*}\left(\sum_{i=0}^{n}\alpha_ix^i\right)=\sum_{i=0}^{n}\tau(\alpha_i)t^i$ es, también, un isomorfismo.

\subsubsection*{\color{teal} Comentarios del lema 5.3:}

Si $F$ y $F'$ son campos y $\tau:F\to F'$ es un isomorfismo, el lema 5.3 asegura que la factorización de $f(x)\in F[x]$ coincide exactamente con la de $\tau^*(f(x))=f'(t)\in F'[t]$, y viceversa.

\subsection*{\color{purple} Irreducibilidad bajo isomorfismo}

Si $F$ y $F'$ son campos y $\tau:F\to F'$ es un isomorfismo, entonces $f(x)\in F[x]$ es irreducible sobre $F$ si y solo si $\tau^*(f(x))\in F'[t]$ es irreducible sobre $F'$. 

\subsection*{\color{blue} Lema 5.4:}

Si $F$ y $F'$ son campos, $\tau:F\to F'$ es un isomorfismo y $f(x)\in F[x]$, entonces existe un isomorfismo $\tau^{**}:F[x]/(f(x))\to F'[t]/(\tau(f(x)))$ tal que si $\alpha\in F$ entonces $\tau^{**}(\alpha)=\tau(\alpha)$, donde $\alpha\approx \alpha+(f(x))$.

\subsection*{\color{red} Teorema 5I:}

Si $F$ es un campo, $p(x)\in F[x]$ es irreducible sobre $F$, $v$ es una raíz de $p(x)$, $F'$ es un campo y $\tau:F\to F'$ es un isomorfismo, entonces existe un isomorfismo $\sigma: F(v)\to F'(w)$, donde $w$ es una raíz de $\tau^*(p(x))$, y este isomorfismo $\sigma$ puede elegirse de manera que $\sigma(v)=w$ y $\sigma(\alpha)=\tau(\alpha)$ para todo $\alpha\in F$. Es decir, $\sigma$ deja fijos, salvo el isomorfismo $\tau$, a los elementos de $F$.

\subsubsection*{\color{red} Corolario del teorema 5I:}

Si $F$ es un campo, $p(x)\in F[x]$ es irreducible sobre $F$, con $a$ y $b$ raíces de $p(x)$, entonces existe un isomorfismo $\sigma: F(a)\to F(b)$ tal que $\sigma(a)=b$ y $\sigma(\alpha)=\alpha$ para todo $\alpha\in F$.

\subsection*{\color{red} Teorema 5J (Unicidad de los campos de descomposición):}

Si $F$ y $F'$ son campos isomorfos bajo $\tau: F\to F'$ y se tiene $f(x)\in F[x]$, entonces el campo de descomposición $E$ de $f(x)$ sobre $F$ es isomorfo al campo de descomposición $E'$ de $\tau^{*}\big(f(x)\big)$ sobre $F'$, y ese isomorfismo mapea a todo $\alpha\in F$ en $\tau(\alpha)\in F'$.

\subsubsection*{\color{teal} Nota del teorema 5J}

En particular, si $F=F'$ y $\tau$ es el isomorfismo identidad de $F$, el teorema 5J asegura que si $E$ y $E'$ son dos campos de descomposición de $f(x)\in F[x]$ sobre $F$, entonces $E\approx E'$ con un isomorfismo que deja fijos a los elementos de $F$.

\newpage
\section{Construcción con regla y compás}

Un número $\alpha\in\mathbb{R}$ es construíble si es posible crear un segmento de largo $\alpha$ solamente haciendo uso de recta y compás.

\subsection*{\color{purple} Cerraduras de los números construíbles}

Si $\alpha$ y $\beta$ son construíbles, entonces $\alpha\pm\beta$, $\alpha\beta$, $\alpha/\beta$ y $\sqrt{\alpha}$ son contruíbles.

\subsection*{\color{violet} Plano de un subcampo de $\mathbb{R}$}

Si $F$ es un subcampo de $\mathbb{R}$, se dice que $F\times F$ es el plano de $F$.

\subsection*{\color{purple} Ecuación de una recta por dos puntos}

Si $F$ es un subcampo de $\mathbb{R}$, entonces la ecuación de la recta en $\mathbb{R}\times\mathbb{R}$ que pasa por $(a_0,b_0)$ y $(a_1,b_1)$ en $F\times F$ es $\alpha x+\beta y+\gamma=0$, con los parámetros $\alpha,\beta,\gamma\in F$ dados por
\begin{align*}
\alpha=\frac{b_1-b_0}{a_1-a_0}, && \beta=-1, && \gamma=b_1-a_1\frac{b_1-b_0}{a_1-a_0}.
\end{align*}

\subsection*{\color{purple} Ecuación de una circunferencia}

Si $F$ es un subcampo de $\mathbb{R}$, entonces la ecuación de la circunferencia en $\mathbb{R}\times\mathbb{R}$ con centro en $(h,k)$ en $F\times F$ y radio $r\in F$ es $x^2+y^2+\alpha x+\beta y+\gamma=0$, con los parámetros $\alpha,\beta,\gamma\in F$ dados por
\begin{align*}
\alpha=-2h, && \beta=-2k, && \gamma=h^2+k^2-r^2.
\end{align*}

\subsection*{\color{purple} Intersección de dos rectas en $F$}

El punto de intersección de dos rectas distintas no paralelas formadas (cada una) por puntos del plano de $F\subseteq\mathbb{R}$, es a su vez un punto en el plano de $F$.

\subsection*{\color{purple} Intersección de una recta y una circunferencia en $F$}

Si $F$ es un subcampo de $\mathbb{R}$, entonces el o los puntos de intersección entre una recta que pasa por puntos del plano de $F$ y una circunferencia con centro en el plano de $F$ y radio en $F$, está o están (si existen) en el plano de $F$ o en el plano de $F(\sqrt{\delta})$, donde $\delta\in F^{+}$.

\subsection*{\color{purple} Intersección de dos circunferencias en $F$}

Si $F$ es un subcampo de $\mathbb{R}$, entonces el o los puntos de intersección entre dos circunferencias con centros en el plano de $F$ y radios en $F$, está o están (si existen) en el plano de $F$ o en el plano de $F(\sqrt{\delta})$, donde $\delta\in F^{+}$.

\subsection*{\color{red} Teorema $5\alpha$:}

Un número $\alpha\in\mathbb{R}$ es construíble si y solo si existe una cantidad finita de números $\lambda_1,\ldots,\lambda_n\in\mathbb{R}^+$ tales que $\lambda_1^2\in\mathbb{Q}$ y $\lambda_i^2\in \mathbb{Q}(\lambda_1,\ldots,\lambda_{i-1})$ para $i\in\{1,2,\ldots,n\}$, para los cuales $\alpha\in \mathbb{Q}(\lambda_1,\ldots,\lambda_n)$.

\subsubsection*{\color{red} Corolario 1 del teorema $5\alpha$:}

Un número $\alpha\in\mathbb{R}$ es construíble si y solo si existe una cantidad finita de números $\lambda_1,\ldots,\lambda_n\in\mathbb{R}^+$ tales que $[\mathbb{Q}(\lambda_1):\mathbb{Q}]\leq2$ y $[\mathbb{Q}(\lambda_1,\ldots,\lambda_{i}):\mathbb{Q}(\lambda_1,\ldots,\lambda_{i-1})]\leq2$ para $i\in\{1,2,\ldots,n\}$.

\subsubsection*{\color{red} Corolario 2 del teorema $5\alpha$:}

Si un número $\alpha\in\mathbb{R}$ es construíble, entonces $\alpha$ pertenece a una extensión finita de $\mathbb{Q}$ cuyo grado sobre $\mathbb{Q}$ es una potencia de $2$.

\subsubsection*{\color{red} Corolario 3 del teorema $5\alpha$:}

Si $\alpha\in\mathbb{R}$ satisface un polinomio irreducible de grado $k$ sobre $\mathbb{Q}$, donde $k$ no es una potencia de $2$, entonces $\alpha$ no es construíble.

\subsection*{\color{red} Teorema $5\beta$ (sobre raíces racionales):}

Si $f(x)=\sum_{k=0}^n a_kx^k\in\mathbb{Z}[x]$ con $a_n$ y $a_0$ distintos de 0 es tal que $f(p/q)=0$ para algún $p/q\in\mathbb{Q}$ (con $p$ y $q$ primos relativos), entonces $p\mid a_0$ y $q\mid a_n$.

\subsubsection*{\color{teal} Aplicaciones}

Estos resultados pueden ser utilizados para demostrar que es imposible trisectar un ángulo de 60 grados, que es imposible construir un cubo cuyo volumen duplica al de otro cubo, es imposible construir un heptágono regular, entre otros.

\newpage
\section{Más sobre raíces}

\subsection*{\color{violet} Derivada de un polinomio}

Si $D$ es un dominio entero y $f(x)=\displaystyle\sum_{k=0}^n a_kx^k\in D[x]$, entonces su derivada es $f'(x)=\displaystyle\sum_{k=1}^n ka_kx^{k-1}$.

\subsection*{\color{violet} Característica de un dominio entero}

La característica de un dominio entero es el menor entero no negativo que anula a todos sus elementos.

\subsubsection*{\color{teal} Característica cero}

Un dominio entero $D$ es de característica 0 si la relación $ma=0$ para $a\neq 0$ en $D$ y $m\in\mathbb{Z}$ solo puede ocurrir si $m=0$.

\subsubsection*{\color{teal} Característica finita}

Un dominio entero $D$ es de característica finita si existe $p\in\mathbb{Z}^{+}$ tal que $pa=0$ para todo $a\in D$. En este caso, $p$ debe ser un número primo.

\subsubsection*{\color{teal} Propiedades de anillos de polinomios de distintas característica}

Si $F$ es un campo de característica $p$ (no cero), la derivada del polinomio $x^p$ en $F[x]$ es $px^{p-1}=0$. Es decir, en un campo de característica finita no necesariamente  un polinomio de derivada nula es un polinomio constante. Sin embargo, si $f(x)\in F[x]$ es un polinomio tal que $f(x^p)$ tiene derivada nula en $F[x]$, este sí debe ser constante. Por su parte, si $F$ es un campo de característica cero y $f(x)\in F[x]$ tiene derivada nula, entonces el polinomio $f(x)$ debe ser constante.

\subsection*{\color{blue} Lema 5.5:}

Si $D$ es un dominio entero con $f(x),g(x)\in D[x]$ y $\alpha\in D$, entonces:
\begin{itemize}
\item $[f(x)+g(x)]'=f'(x)+g'(x)$;
\item $[\alpha f(x)]'=\alpha f'(x)$;
\item $[f(x)g(x)]'=f'(x)g(x)+f(x)g'(x)$.
\end{itemize}

\subsection*{\color{purple} Preservación de factores en extensión}

Si $F$ es un campo, $K$ es una extensión de $F$, y $f(x),g(x)\in F[x]$ tienen un factor común no constante en $K[x]$, entonces tienen también un factor común no constante en $F[x]$.

\subsection*{\color{blue} Lema 5.6:}

Si $F$ es un campo, $f(x)\in F[x]$ tiene una raíz múltiple (con multiplicidad) si y solo si $f(x)$ y $f'(x)$ tienen un factor común no constante.

\subsubsection*{\color{blue} Corolario 1 del lema 5.6:}

Si $F$ es un campo y $f(x)\in F[x]$ es irreducible sobre $F$, entonces:
\begin{itemize}
\item Si la característica de $F$ es cero, entonces $f(x)$ no tiene raíces múltiples.

\item Si la característica de $F$ es distinta de cero, entonces $f(x)$ tiene una raíz múltiple solo si es de la forma $f(x)=g(x^p)$. Esto también nos dice que si la derivada de $f(x)$ es nula se tiene que $f(x)$ tiene raíces múltiples, y viceversa.
\end{itemize}

\subsubsection*{\color{blue} Corolario 2 del lema 5.6 (útil en campos finitos):}

Si $F$ es un campo de característica $p\neq 0$, entonces $x^{p^n}-x\in F[x]$ con $n\geq 1$ tiene raíces distintas.

\subsubsection*{\color{teal} Comentarios sobre los corolarios al lema 5.6}

\begin{itemize}
\item El corolario 1 del lema 5.6 no excluye la posibilidad de que, si la característica de $F$ es distinta de cero, un polinomio irreducible tenga raíces múltiples. Por ejemplo, sea $F_0$ un campo de característica 2 y sea $F=F_0(x)$. Considérese el polinomio $t^2-x\in F[t]$ y nótese que es irreducible sobre $F$, pues de lo contrario $[p(x)]^2=x[q(x)]^2$, lo cual implica que un polinomio de grado par es igual a uno de grado impar. Por otro lado, la derivada de $t^2-x$ en $F[t]$ es $2t$, que como la característica es 2, es nulo. Por corolario 1, $t^2-x$ tiene raíces múltiples.

\item Las implicaciones de distinguir entre campos de característica 0 y de característica positiva son importantes: permiten la posibilidad de polinomios irreducibles con raíces múltiples en el caso de los campos de característica positiva, lo cual es imposible en los campos de característica 0. En esta presentación, se asumirá que todos los campos son de característica 0.
\end{itemize}

\subsection*{\color{violet} Extensiones simples}

Una extensión $K$ de $F$ es simple si $K=F(\alpha)$ para algún $\alpha\in K$.

\subsection*{\color{violet} Elementos y extensiones separables, y campos perfectos}

Si $F$ es un campo y $K$ es una extensión de $F$, entonces $a\in K$ es separable sobre $F$ si satisface a un polinomio en $F[x]$ sin raíces múltiples. Se dice que $K$ es separable sobre $F$ si todos sus elementos son separables sobre $F$. Un campo $F$ es perfecto si todas sus extensiones finitas son separables sobre $F$.

\subsection*{\color{purple} Las extensiones finitas separables son simples}

Si $F$ es un campo y $K$ es una extensión finita y separable de $F$, entonces también es simple. El teorema a continuación muestra que todas las extensiones finitas de los campos de característica cero son simples. Además, las extensiones finitas y separables de los campos de característica positiva son simples.

\subsection*{\color{red} Teorema 5K:}

Si $F$ es un campo de característica cero con $a$ y $b$ son algebraicos sobre $F$. Entonces, existe $c\in F(a,b)$ tal que $F(a,b)=F(c)$.

\subsubsection*{\color{red} Corolario del teorema 5K:}

Toda extensión finita de un campo de característica cero es simple.

\newpage
\section{Elementos de la teoría de Galois}

\subsection*{\color{violet} Automorfismos de campos}

Si $K$ es un campo, un automorfismo de $K$ es un isomorfismo $\sigma:K\to K$.

\subsection*{\color{red} Teorema 5L:}

Si $K$ es un campo y $\sigma_1,\ldots,\sigma_n$ son automorfismos distintos de $K$, entonces es imposible encontrar elementos no todos nulos $a_1,\ldots,a_n\in K$ tales que $\displaystyle\sum_{i=1}^n a_i\sigma_i(k)=0$ para todo $k\in K$.

\subsection*{\color{violet} Campos fijados}

Si $K$ es un campo y $G$ es un grupo de automorfismos de $K$, entonces el campo fijado por $G$ (el cual es subconjunto de $K$) se define como $\{k\in K\mid \forall\sigma\in G, \sigma(k)=k\}$.

\subsubsection*{\color{teal} Nota sobre la definición de campos fijados}

Esta definición hace sentido si en vez del grupo $G$ se usa cualquier subconjunto de $\mathbb{A}(K)$, el cual es el conjunto de todos los automorfismos en $K$. De hecho, es posible demostrar que si $B\subseteq\mathbb{A}(K)$ entonces el conjunto fijado por $B$ es igual al conjunto fijado por $(B)$. 

\subsection*{\color{blue} Lema 5.7:}

Si $K$ es un campo y $G\subseteq\mathbb{A}(K)$, entonces el subconjunto de $K$ fijado por $G$ es un subcampo de $K$.

\subsection*{\color{violet} Grupo de automorfismos relativo}

Sea $F$ un campo y $K$ una extensión de $F$. Entonces, el grupo de automorfismos de $K$ relativo a $F$ se denota por $G(K,F)$, y está definido por $G(K,F)=\{\sigma\in\mathbb{A}(K)\mid \forall\alpha\in F, \sigma(\alpha)=\alpha\}$.

\subsection*{\color{blue} Lema 5.8:}

Si $F$ un campo y $K$ es una extensión de $F$, entonces $G(K,F)$ es un subgrupo de $\mathbb{A}(K)$.

\subsubsection*{\color{teal} Nota sobre los últimos resultados}

Si $K$ es un campo de característica cero, entonces es un campo infinito. Puesto que $\mathbb{Q}$ es el campo infinito más pequeño, todo campo infinito contiene una copia isomórfica de $\mathbb{Q}$. Entonces, todo campo fijado por cualquier grupo de automorfismos de $K$ contiene a $\mathbb{Q}$, y por lo tanto todo número racional es invariante en la acción de cualquier automorfismo de un campo de característica cero. Por otra parte, en general, $G(K,F)$ no siempre es cíclico ni abeliano. 

\subsection*{\color{teal} Algunos ejemplos de grupos de automorfismos}

\subsubsection*{\color{teal} Los complejos relativo a los reales}

Considérese $G(\mathbb{C},\mathbb{R})$. Si $\sigma:\mathbb{C}\to\mathbb{C}$ es un automorfismo que deja fijo a $\mathbb{R}$, entonces $\sigma(i)^2=\sigma(i^2)=\sigma(-1)=-1$, lo cual implica que $\sigma(i)$ puede tomar el valor de $i$ o $-i$. Si $\sigma_1(i)=i$, entonces en general $\sigma_1(a+bi)=\sigma_1(a)+\sigma_1(b)\sigma_1(i)=a+bi$. Es decir, $\sigma_1$ es la identidad. Si $\sigma_2(i)=-i$, entonces en general $\sigma_2(a+bi)=\sigma_2(a)+\sigma_2(b)\sigma_2(i)=a-bi$. Es decir, $\sigma_2$ es la conjugación. Entonces, $G(\mathbb{C},\mathbb{R})=\{I_{\mathbb{C}},\overline{\cdots}\}$. Como los únicos elementos que quedan fijos bajo todo elemento de $G(\mathbb{C},\mathbb{R})$ son los reales, entonces el campo fijado por $G(\mathbb{C},\mathbb{R})$ es $\mathbb{R}$ mismo.

\subsubsection*{\color{teal} Los racionales adjuntos a $\sqrt[3]{2}$ relativo a los racionales}

Considérese $G(\mathbb{Q}(\sqrt[3]{2}),\mathbb{Q})$. Si $\sigma:\mathbb{Q}(\sqrt[3]{2})\to\mathbb{Q}(\sqrt[3]{2})$ es un automorfismo que deja fijo a $\mathbb{Q}$, entonces $\sigma(\sqrt[3]{2})^3=\sigma(2)=2$. Hay tres números complejos que cumplen esto, de los cuales solo $\sqrt[3]{2}\in\mathbb{Q}(\sqrt[3]{2})$. Por lo tanto, solo existe la posibilidad de $\sigma(\sqrt[3]{2})=\sqrt[3]{2}$. Entonces en general, 
$$\sigma(a+b\sqrt[3]{2}+c\sqrt[3]{2}^2)=\sigma(a)+\sigma(b)\sigma(\sqrt[3]{2})+\sigma(c)\sigma(\sqrt[3]{2})^2=a+b\sqrt[3]{2}+c\sqrt[3]{2}^2.$$
Esto es, $\sigma$ fija a todo elemento de $\mathbb{Q}(\sqrt[3]{2})$, y por lo tanto es el automorfismo identidad, y es el único. Es decir, $G(\mathbb{Q}(\sqrt[3]{2}),\mathbb{Q})=\{I_{\mathbb{Q}(\sqrt[3]{2})}\}$. Además, el campo fijado por $G(\mathbb{Q}(\sqrt[3]{2}),\mathbb{Q})$ es $\mathbb{Q}(\sqrt[3]{2})$ en su totalidad.

\subsubsection*{\color{teal} Los racionales adjuntos a $e^{2\pi i/5}$ relativo a los racionales}

Sea $\omega=e^{2\pi i/5}$, y considérese $G(\mathbb{Q}(\omega),\mathbb{Q})$. Dado que es una raíz quinta de la unidad, es claro que $\omega^5=1$ y satisface al polinomio $x^4+x^3+x^2+x+1\in\mathbb{Q}[x]$. Es claro que este es irreducible sobre $\mathbb{Q}$, y por lo tanto $[\mathbb{Q}(\omega):\mathbb{Q}]=4$ y todo elemento en $\mathbb{Q}(\omega)$ es de la forma $\alpha_0+\alpha_1\omega+\alpha_2\omega^2+\alpha_3\omega^3$, con $\alpha_0,\alpha_1,\alpha_2,\alpha_3\in\mathbb{Q}$. Ahora bien, si $\sigma:\mathbb{Q}(\omega)\to\mathbb{Q}(\omega)$ es un automorfismo que deja fijo a $\mathbb{Q}$, entonces $\sigma(\omega)^5=\sigma(1)=1$. Pero nótese que $\sigma(\omega)\neq1$, puesto que $\sigma$ es sobreyectivo y $\sigma(1)=1$. Entonces podemos hacer $\sigma_k(\omega)=\omega^k$ para $k\in\{1,2,3,4\}$. Es claro que $\sigma_1$ es el automorfismo identidad, y no es difícil demostrar que $\sigma_2^2=\sigma_4$, $\sigma_2^3=\sigma_3$, y $\sigma_2^4=\sigma_1$. Entonces, $G(\mathbb{Q}(\omega),\mathbb{Q})=\{\sigma_1,\sigma_2,\sigma_3,\sigma_4\}$ es un grupo cíclico de orden 4. Además, tampoco es difícil demostrar que el campo fijado en este caso es $\mathbb{Q}$ mismo. Por su parte, también es fácil ver que $\sigma_4^2=\sigma_1$, y entonces $\{\sigma_1,\sigma_4\}$ es un subgrupo cíclico de orden 2 de $G(\mathbb{Q}(\omega),\mathbb{Q})$. Por su parte, este tiene como campo fijado a todos los elementos de la forma $\alpha_0+\alpha_2(\omega^2+\omega^3)$, el cual es el campo $\mathbb{Q}(\omega^2+\omega^3)$, y cumple $[\mathbb{Q}(\omega^2+\omega^3):\mathbb{Q}]=2$.

\subsection*{\color{red} Teorema 5M:}

Si $F$ es un campo y $K$ es una extensión finita de $F$, entonces $o\big(G(K,F)\big)\leq[K:F]$.

\subsubsection*{\color{teal} Permutaciones en las funciones racionales de varias variables}

Si $F$ es un campo, $\sigma\in S_n$ y $r(x_1,\ldots,x_n)\in F(x_1,\ldots,x_n)$, entonces abreviando la notación es posible ver a $\sigma$ actuando sobre $F(x_1,\ldots,x_n)$:
$$\sigma\big(r(x_1,\ldots,x_n)\big)=r(x_{\sigma(1)},\ldots,x_{\sigma(n)}).$$
De esta forma, $\sigma$ es un automorfismo de $F(x_1,\ldots,x_n)$.

\subsection*{\color{violet} Campo de las funciones racionales simétricas}

Si $F$ es un campo, el campo de las funciones racionales simétricas $S$ sobre $F$ es el subcampo de $F(x_1,\ldots,x_n)$ fijado por $S_n$. Es decir, $r(x_1,\ldots,x_n)\in S$ si y solo si $r(x_1,\ldots,x_n)=r(x_{\sigma(1)},\ldots,x_{\sigma(n)})$ para todo $\sigma\in S_n$.

\subsection*{\color{violet} Funciones simétricas elementales}

Las funciones simétricas elementales de $x_1,\ldots,x_n$ son:
\begin{align*}
a_1(x_1,\ldots,x_n)=\sum_{i=1}^n x_i; && a_2(x_1,\ldots,x_n)=\sum_{i<j}^n x_ix_j; && \cdots; && a_n(x_1,\ldots,x_n)=\prod_{i=1}^n x_i.
\end{align*}
En efecto, estos polinomios pertenecen al campo de las funciones racionales simétricas.

\newpage
\subsubsection*{\color{teal} Polinomios simétricos de dos variables}

Si $n=2$, entonces $a_1=x_1+x_2$ y $a_2=x_1x_2$. Nótese que $x_1$ y $x_2$ son las raíces de $t^2-a_1t+a_2$. 

\subsubsection*{\color{teal} Polinomios simétricos de tres variables}

Si $n=3$, entonces $a_1=x_1+x_2+x_3$, $a_2=x_1x_2+x_3x_1$ y $a_3=x_1x_2x_3$. Nótese que $x_1$, $x_2$ y $x_3$ son las raíces de $t^3-a_1t^2+a_2t-a_3$.

\subsubsection*{\color{teal} Polinomios simétricos de cuatro variables}

Si $n=3$, entonces $a_1=x_1+x_2+x_3+x_4$, $a_2=x_1x_2+x_2x_3+x_3x_4+x_4x_1$, $a_3=x_1x_2x_3+x_2x_3x_4+x_3x_4x_1+x_4x_1x_2$. Nótese que $x_1$, $x_2$, $x_3$ y $x_4$ son las raíces de $t^4-a_1t^3+a_2t^2-a_3t+a_4$.

\subsubsection*{\color{purple} Polinomios simétricos de $n$ variables}

En general, $x_1,\ldots,x_n$ son las raíces de $\displaystyle\sum_{i=0}^n (-1)^ia_it^{n-i}$ con $a_0=1$.

\subsection*{\color{red} Teorema 5N:}

Si $F$ es un campo y $n\in\mathbb{Z}^+$, entonces:
\begin{itemize}
\item $[F(x_1,\ldots,x_n):S]=n!$;

\item $G\big(F(x_1,\ldots,x_n), S\big)=S_n$;

\item (Teorema de polinomios simétricos): Si $a_1,\ldots,a_n$ son las funciones simétricas elementales en las variables $x_1,\ldots,x_n$, entonces $S=F(a_1,\ldots,a_n)$. Esto es, toda función racional simétrica de $n$ variables es una función racional evaluada en las $n$ funciones simétricas elementales. O, puede enunciarse de manera que todo polinomio simétrico en $n$ variables es un polinomio evaluado en las $n$ funciones simétricas elementales;

\item $F(x_1,\ldots,x_n)$ es el campo de descomposición sobre $S=F(a_1,\ldots,a_n)$ del polinomio definido por $\displaystyle\sum_{i=0}^n (-1)^ia_it^{n-i}\in F(a_1,\ldots,a_n)[t]$ con $a_0=1$.
\end{itemize}

\subsubsection*{\color{teal} Comentarios sobre el teorema 5N}

En los comentarios al teorema 5H se mencionó que dado un $n\in\mathbb{Z}^+$ es posible contruir un campo y un polinomio de grado $n$ sobre dicho campo, cuyo campo de descomposición es de grado máximo ($n!$). El teorema 5N explícitamente provee un ejemplo, dado un campo $F$.

\subsection*{\color{violet} Extensión normal}

Si $F$ es un campo, se dice que $K$ es una extensión normal de $F$ si esta es una extensión finita de $F$ y además $F$ es el campo fijado por $G(K,F)$.

\subsubsection*{\color{teal} Ejemplos de extensiones normales}

De los ejemplos vistos respecto a grupos de automorfismos, es claro que $\mathbb{C}$ es  extensión normal de $\mathbb{R}$; $\mathbb{Q}(\sqrt[3]{2})$ no es extensión normal de $\mathbb{Q}$; y $\mathbb{Q}(e^{2\pi i/5})$ es extensión normal de $\mathbb{Q}$.

\subsubsection*{\color{purple} Propiedad de extensiones normales}

Si $F$ es un campo, $K$ es una extensión normal de $F$ y $k\in K\sim F$, entonces $k$ no queda invariante por algún automorfismo elemento de $G(K,F)$.

\subsection*{\color{red} Teorema 5O:}

Sea $F$ un campo de característica cero, $K$ una extensión normal de $F$, sea $H$ un subgrupo de $G(K,F)$ y sea $K_H$ el subcampo de $K$ fijado por $H$. Entonces:
\begin{itemize}
\item $[K:K_H]=o(H)$;

\item $H=G(K,K_H)$.
\end{itemize}
En particular, cuando $H=G(K,F)$, se tiene que $[K:F]=o\big(G(K,F)\big)$.

\subsection*{\color{blue} Lema 5.9:}

Si $F$ es un campo, $K$ es el campo de descomposición sobre $F$ de un polinomio $f(x)\in F[x]$, si $p(x)\in F[x]$ es irreducible y es factor de $f(x)$, y si $\alpha_1,\ldots,\alpha_r$ son las raíces de $p(x)$, entonces para cada $\alpha_i$ existe $\sigma_i\in G(K,F)$ tal que $\sigma_i(\alpha_1)=\alpha_i$.

\subsection*{\color{red} Teorema 5P:}

Si $F$ es un campo de característica cero, entonces $K$ es una extensión normal de $F$ si y solo si $K$ es el campo de descomposición de un polinomio sobre $F$.

\subsection*{\color{violet} Grupo de Galois de un polinomio}

Si $F$ es un campo y $K$ es el campo de descomposición de $f(x)\in F[x]$ sobre $F$, el grupo de Galois de $f(x)$ sobre $F$ es $G(K,F)$.

\subsubsection*{\color{violet} Nota sobre los grupos de Galois}

El grupo de Galois de $f(x)$ sobre $F$ puede considerarse como un grupo de permutaciones de las raíces de $f(x)$. Si $\alpha$ es una raíz de $f(x)$ y $\sigma\in G(K,F)$, entonces $\sigma(\alpha)$ es también una raíz de $f(x)$.

\subsection*{\color{red} Teorema 5Q (Teorema fundamental de la teoría de Galois):}

Si $F$ es un campo, $K$ es el campo de descomposición de $f(x)\in F[x]$ sobre $F$, y $G(K,F)$ es el grupo de Galois de $f(x)$ sobre $F$. Entonces, para todo $T$ subcampo de $K$ y para todo $H$ subgrupo de $G(K,F)$ existe una correspondencia biyectiva entre la colección de subcampos de $K$ que contienen a $F$ y los subgrupos de $G(K,F)$, tal que:

\begin{itemize}

\item $T=K_{G(K,T)}$;

\item $H=G(K,K_H)$;

\item $[K:T]=o\big(G(K,T)\big)$ y $[T:F]=i_{G(K,F)}\big(G(K,T)\big)$;

\item $T$ es una extensión normal de $F$ si y solo si $G(K,T)$ es un subgrupo normal de $G(K,F)$;

\item Si $T$ es una extensión normal de $F$, entonces $G(T,F)\approx G(K,F)/G(K,T)$.

\end{itemize}

\newpage

\section{Solubilidad por radicales}

Si $F$ es un campo, entonces $p(x)\in F[x]$ es soluble por radicales si existe una colección finita de campos 
$$F_1=F(\omega_1),F_2=F_1(\omega_2),\ldots,F_k=F_{k-1}(\omega_k)=F(\omega_1,\ldots,\omega_k),$$
tales que $\omega_1^{r_1}\in F,\omega_2^{r_2}\in F_1,\ldots,\omega_k^{r_k}\in F_{k-1}$, para $r_1,r_2,\ldots,r_k\in\mathbb{Z}^+$, y además $F_k$ contiene al campo de descomposición de $p(x)$ sobre $F$.

\subsection*{\color{purple} Normalidad de $F_k$}

Si $F$ es un campo con $p(x)\in F[x]$ soluble por radicales, entonces $F_k$ es una extensión normal de $F$.

\subsection*{\color{violet} Polinomio general}

Si $F$ es un campo, entonces $p(x)=\displaystyle\sum_{i=0}^n a_ix^{n-i}\in F[x]$ con $a_0=1$ es el polinomio general de grado $n$ sobre $F$, y $p(x)$ es un polinomio particular de $F(a_1,\ldots,a_n)$.

\subsection*{\color{violet} Grupos solubles}

Un grupo $G$ es soluble si existe una cadena finita de subgrupos tales que
$$(e)=N_k\subseteq N_{k-1}\subseteq\cdots\subseteq N_2\subseteq N_1\subseteq N_0=G,$$
donde $N_k$ es subgrupo normal de $N_{k-1}$ y además $N_{k-1}/N_k$ es abeliano, para cada $k$.

\subsubsection*{\color{teal} Ejemplos de grupos solubles}

\begin{itemize}

\item Todo grupo abeliano es soluble, con $N_0=G$ y $N_1=(e)$.

\item $S_3$ es soluble.

\item $S_4$ es soluble.

\item $S_n$ con $n\geq 5$ no es soluble.

\end{itemize}

\subsection*{\color{violet} Subgrupo conmutador}

Dado un grupo $G$ con $g_1,g_2\in G$, el conmutador de $g_1$ y $g_2$ es $g_1^{-1}g_2^{-1}g_1g_2$, y el subgrupo conmutador de $G$, denotado $G'$, es el subgrupo de $G$ generado por todos los conmutadores de $G$. Este es un subgrupo normal de $G$, y además $G/G'$ es un grupo abeliano. Además, $G'$ es el subgrupo normal de $G$ más pequeño cuyo cociente es abeliano. Es decir, si $M$ es normal en $G$ y $G/M$ es abeliano, $G'\subseteq M$.

\subsection*{\color{violet} Segundo subgrupo conmutador}

Si $G$ es un grupo, $G''$ es el subgrupo conmutador de $G'$. Resulta que $G''$ es un subgrupo normal de $G'$ y de $G$. Además, $G'/G''$ es abeliano.

\subsection*{\color{violet} $n$-avo subgrupo conmutador}

Si $G$ es un grupo, $G^{(n)}$ es el subgrupo conmutador de $G^{(n-1)}$. Resulta que $G^{(n)}$ es un subgrupo normal de $G^{(k)}$ para todo $1\leq k\leq n-1$, y de $G$. Además, $G^{(n-1)}/G^{(n)}$ es abeliano.

\subsection*{\color{blue} Lema 5.10:}

Un grupo $G$ es soluble si y solo si existe $k\in\mathbb{N}$ tal que $G^{(k)}=(e)$.

\subsubsection*{\color{blue} Corolario del lema 5.10:}

Toda imagen homomórfica de un grupo soluble es un grupo soluble.

\subsection*{\color{blue} Lema $5.\alpha$:}

Si $G$ es un grupo y $N$ es un subgrupo normal de $G$, entonces $N'$ es un subgrupo normal de $G$.

\subsection*{\color{blue} Lema 5.11:}

Si $n\geq 5$ y $k\in\mathbb{Z}^+$, entonces $S_n^{(k)}$ contiene a todos los $3$-ciclos de $S_n$.

\subsection*{\color{red} Teorema 5R:}

Si $n\geq 5$, entonces $S_n$ no es soluble.

\subsection*{\color{blue} Lema 5.12:}

Sea $F$ un campo de característica cero que contiene a las $n$-ésimas raíces de la unidad, sea $a\in F$ no nulo, y sea $K$ el campo de descomposición de $x^n-a\in F[x]$ sobre $F$. Entonces,

\begin{itemize}
\item $K=F(u)$, donde $u$ es cualquier raíz de $x^n-a$;

\item El grupo de Galois de $x^n-a$ sobre $F$ es abeliano.
\end{itemize}

\subsubsection*{\color{teal} Nota del lema 5.12}

Este lema nos dice que si $F$ contiene a todas las $n$-ésimas raíces de la unidad, entonces al adjuntar una raíz $u$ de $x^n-a$ se obtiene el campo de descomposición sobre $F$ del polinomio $x^n-a$, es decir, $F(u)$ es una extensión normal de $F$.

\subsection*{\color{red} Teorema 5S:}

Si $F$ es un campo de característica cero que contiene a todas las $n$-ésimas raíces de la unidad y $p(x)\in F[x]$ es un polinomio soluble por radicales, entonces el grupo de Galois de $p(x)$ sobre $F$ es un grupo soluble.

\subsection*{\color{red} Teorema $5\gamma$:}

Si el grupo de Galois de un polinomio es soluble, entonces el polinomio es soluble por radicales.

\subsection*{\color{red} Teorema $5\delta$:}

Dado un campo $F$ cualquiera, entonces $p(x)\in F[x]$ es un polinomio soluble por radicales si y solo si el grupo de Galois de $p(x)$ sobre $F$ es un grupo soluble.

\subsection*{\color{red} Teorema 5T (Teorema de Abel):}

El polinomio general de grado $n\geq 5$ no es soluble por radicales.

\newpage
\section*{Los grupos de Galois sobre $\mathbb{Q}$}

\subsection*{\color{red} Teorema 5U (Infinitos ejemplos de alcanzar la cota superior):}

Si $q(x)\in\mathbb{Q}[x]$ es irreducible, $p=\gr(q)$ es un número primo, y $q(x)$ posee exactamente dos raíces no reales en $\mathbb{C}$, entonces el grupo de Galois de $q(x)$ sobre $\mathbb{Q}$ es $S_p$. Además, si $K$ es el campo de descomposición de $q(x)$ sobre $\mathbb{Q}$, entonces $[K:\mathbb{Q}]=o\big(G(K,\mathbb{Q})\big)=o(S_p)=p!$.

\subsection*{\color{teal} Ejemplo 1: $x^3-2$}

El polinomio $x^3-2\in\mathbb{Q}$ es irreducible, y posee exactamente una raíz real y dos raíces complejas no reales. Entonces el grupo de Galois de $x^3-2$ es $S_3$, y por lo tanto $[E:\mathbb{Q}]=o(S_3)=3!=6$ (donde $E$ es el campo de descomposición de $x^3-2$ sobre $\mathbb{Q}$). Se sabía con anterioridad que el campo de descomposición de $x^3-2$ sobre $\mathbb{Q}$ es $\mathbb{Q}(\omega)$, donde $\omega^3=1$ y $\omega\neq 1$. En efecto, es fácil comprobar que se cumple $[\mathbb{Q}(\omega):\mathbb{Q}]=6=3!=o(S_3)$.

\subsection*{\color{teal} Ejemplo 2: $2x^5-10x+5$}

El polinomio $2x^5-10x+5\in\mathbb{Q}[x]$ es irreducible, y posee exactamente tres raíces reales y dos raíces complejas no reales. El grupo de Galois del polinomio es entonces $S_5$, y entonces $q(x)$ no es soluble por radicales., y por lo tanto $[E:\mathbb{Q}]=o(S_5)=5!=120$ (donde $E$ es el campo de descomposición de $2x^5-10x+5$ sobre $\mathbb{Q}$). Además, por el teorema de Abel, $2x^5-10x+5$ no es soluble por radicales.

\end{document}